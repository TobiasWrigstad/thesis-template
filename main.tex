\documentclass[11pt,titlepage,openright]{book}
\usepackage[utf8]{inputenc}
\usepackage[T1]{fontenc}
\usepackage[british]{babel}
\usepackage{graphicx}
\usepackage[dvipsnames]{xcolor}
\usepackage{libertine}
\renewcommand*\ttdefault{cmtt}

\usepackage[sf]{titlesec}
\usepackage[square,sort,colon,authoryear]{natbib}
%\usepackage[export]{adjustbox}

\usepackage{
  marginnote, % Improved margin notes
  environ,
  ragged2e,   % Justified text in the margin notes
  url,        % For typesetting URLs
  listings,   % Code formatting
  hyperref,   % Links in PDF from TOC, refs, etc.
  lipsum,
  booktabs,
  changepage,
  float,
  dingbat
}

\usepackage[capitalise]{cleveref}
\Crefname{section}{\S}{\S\S}

\usepackage{enumitem}   % Improved lists 
% \setlist{noitemsep} % To make all lists compact
\setlist[itemize]{itemsep=0pt}
\setlist[itemize,1]{label=--}
\setlist[itemize,2]{label=\ensuremath{\triangleright}}
\setlist[itemize,3]{label=\RED{AVOID}}

\usepackage[twoside,labelfont=sf]{caption}

\captionsetup{justification=raggedright,singlelinecheck=false}

\newcommand\myhrulefill[1]{\leavevmode\leaders\hrule height#1\hfill\kern0pt}
\DeclareCaptionFormat{FigFormat}{{\color{black}\myhrulefill{0.5pt}}\\#1#2#3}
\captionsetup[figure]{format=FigFormat}
\captionsetup[table]{format=FigFormat}

\DeclareCaptionFormat{LstFormat}{\textsf{Listing}~\arabic{chapter}.\arabic{listing}:#2#3}
\floatstyle{ruled}
\newfloat{listing}{thp}{lol}[chapter]
\floatname{listing}{Listing}
\captionsetup[listing]{format=LstFormat}

\NewEnviron{MarginNote}[1][0mm]{\marginnote{\footnotesize\justifying\BODY}[#1]}
\newcommand{\Footnote}[2][0mm]{\footnotemark\marginnote{\footnotesize$^{\arabic{footnote})}$~#2}[#1]}

\renewenvironment{figure*}[1][]{%
  \begin{figure}[#1]%
    \checkoddpage%
    \ifoddpage%
      \begin{adjustwidth}{0cm}{-45mm}%%
    \else%
      \begin{adjustwidth}{-45mm}{0cm}%%
    \fi%
    }{%
    \end{adjustwidth}%
  \end{figure}}

\renewenvironment{table*}[1][]{%
  \begin{table}[#1]%
    \checkoddpage%
    \ifoddpage%
      \begin{adjustwidth}{0cm}{-45mm}%%
    \else%
      \begin{adjustwidth}{-45mm}{0cm}%%
    \fi%
    }{%
    \end{adjustwidth}%
  \end{table}}

\renewenvironment{listing*}[1][]{%
  \begin{listing}[#1]%
    \checkoddpage%
    \ifoddpage%
      \begin{adjustwidth}{0cm}{-45mm}%%
    \else%
      \begin{adjustwidth}{-45mm}{0cm}%%
    \fi%
    }{%
    \end{adjustwidth}%
  \end{listing}}

%% == Code =======================================================
\lstnewenvironment{Code}[1][style=std]{\lstset{#1}}{}
\lstnewenvironment{Code_Numbered}[1][style=std,numbers=left]{\lstset{#1}}{}

\renewcommand{\c}[1]{\lstinline[style=std]@#1@}

\lstdefinestyle{std}{
  language=java,
  basicstyle=\small\sf\color{black},
  keywordstyle=\small\sf\bfseries,
  numberstyle=\footnotesize\sf\color{black},
  commentstyle=\small\color{black}\it,
  aboveskip=1ex,
  belowskip=1ex,
  tabsize=2,
  columns=fullflexible,
  xleftmargin=1ex,
  resetmargins=true,
  showstringspaces=false,
  morecomment=[l]{//},
  morecomment=[l]{--},
  morecomment=[s]{/*}{*/},
  escapeinside=@@,
  morekeywords={Frobies},
  moredelim=[is][\textit]{___}{___},
  moredelim=[is][\textbf]{__*}{*__},
  numberbychapter=true
}

\usepackage[activate={true,nocompatibility},final,tracking=true,kerning=true,spacing=true,factor=1100,stretch=10,shrink=10]{microtype}
\usepackage[paper=a4paper,text={13cm,24cm},marginparsep=5mm,marginparwidth=45mm,inner=20mm,twoside]{geometry}

\newcommand{\RED}[1]{\textcolor{red}{#1}}
\newcommand{\ie}{\emph{i.e.,}}
\newcommand{\eg}{\emph{e.g.,}}
\newcommand{\etal}{\emph{et~al.}}

\usepackage{pifont}
\newcommand{\Yes}{\ding{51}}
\newcommand{\No}{\ding{55}}

\renewcommand{\bfdefault}{b}
\clearpage{\pagestyle{empty}\cleardoublepage}

%% You can remove this line if you compile with --synctex=1 (see Makefile)
\synctex=1
\pagestyle{plain}

\begin{document}
\frontmatter
\title{Thesis Template}
\author{Tobias Wrigstad}
\date{\today}

\maketitle

\vspace*{3cm}
\section*{Abstract}

\RED{Replace this with the actual abstract. Obviously.}
Suspendisse luctus leo et porta mattis. In semper, nisi et
suscipit iaculis, leo urna laoreet lacus, ut laoreet lorem tellus
eget dui. Vestibulum eu auctor nisi. Morbi pharetra euismod velit
ac mattis. Maecenas tempor vitae augue ut aliquam. Nunc auctor,
nibh at imperdiet finibus, ex leo semper lacus, ac vehicula quam
nisl condimentum leo.


\tableofcontents
\listoffigures
\listoftables

\mainmatter

\chapter{Introduction}

\lipsum[1]

\begin{MarginNote}[3cm]
  \lipsum[2]

  \RED{Use linebreaks\-to\-insert\-hyphens\-where\-needed\-in the margin.}
\end{MarginNote}

Footnotes\Footnote[-1cm]{Like this one. Note the manual vertical offset in the source code.} will go into the margin. They should be used sparingly.

\begin{itemize}[label=\leftpointright]
\item The \texttt{enumitem} package will help you customise lists
\item Like weird labels like this
  \begin{itemize}
  \item Try \texttt{texdoc enumitem} on your command line if you
    have a local \LaTeX{} installation (and have your mind blown)
  \end{itemize}
\end{itemize}

You can use \texttt{cleveref} to cross-reference in your text.
For chapters, it will produce something like this: \Cref{cha:background}.
For sections, it will produce something like this: \Cref{sec:purpose-goals}.

\lipsum[3]

\section{Purpose and Goals}
\label{sec:purpose-goals}

\lipsum[4-6]

\begin{itemize}
\item FWIW, this is my favourite list label for unordered lists. I
  find bullets too ``clumsy.'' If you do not like it, you can
  override the default settings (see the source).
\item But your thesis is your thesis. And, ofc, don't use words
  like FWIW, ofc, and don't in a thesis text.

  \begin{itemize}
  \item You can nest lists if you want.

    \begin{itemize}
    \item But do not go overboard.
    \end{itemize}

  \end{itemize}
  
\end{itemize}

This is how you typeset descriptions:

\begin{description}
\item[Beach Boys] \lipsum[1]
\item[The Ronettes] \lipsum[2]
\end{description}

Enumerated lists:

\begin{enumerate}
\item One
\item Two
\end{enumerate}

Note that you can resume lists.

\begin{enumerate}[resume]
\item One
\item Two \label{lst:two}
\end{enumerate}

And reference back to individual list items, \eg{}~\cref{lst:two}.

\section{Thesis Outline}
\label{sec:thesis-outline}

\lipsum[7]

\begin{figure}[t]
  \begin{center}
    \vspace*{1em}
    \verb+\includegraphics[width=0.9\linewidth]{image.pdf}+
    \vspace*{1em}
  \end{center}
  \caption{This is a figure, it will be placed top-most of a page (if possible).}
  \label{fig:example-1}
\end{figure}

\begin{figure*}[t]
  \begin{center}
    \vspace*{1em}
    \verb+\includegraphics[width=0.9\linewidth]{image.pdf}+
    \vspace*{1em}
  \end{center}
\caption[Check the source to see how to customise a figure's LoF entry.]{This is a wider figure, it too will be placed top-most of a page (if possible).}
  \label{fig:example-1}
\end{figure*}


\begin{table}[b]
  \caption{This is a table, it will be placed at the bottom of a
    page (if possible). Note that captions for tables are placed
    \emph{above} the table.}
  \begin{center}
    \begin{tabular}{lll}
      \toprule
      $A$ & $B$ & $C$ \\
      \midrule
      $A$ & \multicolumn{2}{c}{$B-C$} \\
      $A$ & \Yes{} & \No{} \\
      \bottomrule
    \end{tabular}
  \end{center}
  
  \label{tab:example-1}
\end{table}

\chapter{Background}
\label{cha:background}

This chapter typically contains a discussion of the related work.
BibTeX is a great way to manage bibliography information. If you
use Emacs, tools like RefTeX will help you insert references into
your code. Given that we are not constrained for space, we can use
citations that look like this: \citep{Acm_Curriculum_2013,
  CWE_193}. To give the text a bit better flow, you can sometimes
work them into the text. Using numbered references is allowed but
it will cause them to be much like Dijkstra's
\citeyearpar{Dijkstra_1968} dreaded GOTO's.

\chapter{Methodology}
\lipsum[1] Code \c{int inline = ++x;} looks like so:

\begin{Code}
public class HelloWorld<T> extends Something<T> {
  public static void main(String[] args) {
    System.out.println("Hello, world!");
  }
}
\end{Code}

You can also get it with line numbers:

\begin{Code_Numbered}
public class HelloWorld<T> extends Something<T> {
  public static void main(String[] args) {
    System.out.println("Hello, world!"); @\label{code:hello}@
  }
}
\end{Code_Numbered}


You can refer to lines like this: \verb+\cref{code:hello}+ (gives you \cref{code:hello}, see source for more details).

Verbatim text is set in a monospaced typewriter font:

\begin{verbatim}
----------,
          |           ASCII art!
          '--> (*)
\end{verbatim}

\lipsum[2-4]

\begin{table}[!h]
  \caption{bar}
  \label{tab:bar}
\end{table}

\begin{listing*}[t]
\begin{Code_Numbered}
public class HelloWorld<T> extends Something<T> {
  public static void main(String[] args) {
    System.out.println("Hello, world!"); @\label{code:hello}@
  }
}
\end{Code_Numbered}
\caption{If you like, you can have code listings inside a \c{listing} float.}
\label{lst:foo}
\end{listing*}

Referencing Listing~\ref{lst:foo}.
Probably better to use cleveref: \cref{lst:foo}.

\chapter{Design}
\lipsum

\chapter{Implementation}
\lipsum

\chapter{Evaluation}
\lipsum

\chapter{Conclusions}
\lipsum

\bibliographystyle{plainnat}
\bibliography{main}

\end{document}

%%% Local Variables: ***
%%% mode: latex ***
%%% TeX-master: "main.tex"  ***
%%% ispell-local-dictionary: "british"  ***
%%% End: ***